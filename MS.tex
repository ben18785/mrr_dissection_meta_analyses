\documentclass[]{article}
\usepackage{lmodern}
\usepackage{amssymb,amsmath}
\usepackage[citestyle=authoryear,natbib=true,backend=bibtex,giveninits=true]{biblatex}
\usepackage{ifxetex,ifluatex}
\usepackage{fixltx2e} % provides \textsubscript
\ifnum 0\ifxetex 1\fi\ifluatex 1\fi=0 % if pdftex
  \usepackage[T1]{fontenc}
  \usepackage[utf8]{inputenc}
\else % if luatex or xelatex
  \ifxetex
    \usepackage{mathspec}
  \else
    \usepackage{fontspec}
  \fi
  \defaultfontfeatures{Ligatures=TeX,Scale=MatchLowercase}
\fi
% use upquote if available, for straight quotes in verbatim environments
\IfFileExists{upquote.sty}{\usepackage{upquote}}{}
% use microtype if available
\IfFileExists{microtype.sty}{%
\usepackage[]{microtype}
\UseMicrotypeSet[protrusion]{basicmath} % disable protrusion for tt fonts
}{}
\PassOptionsToPackage{hyphens}{url} % url is loaded by hyperref
\usepackage[unicode=true]{hyperref}
\hypersetup{
            pdfborder={0 0 0},
            breaklinks=true}
\urlstyle{same}  % don't use monospace font for urls
\usepackage{longtable,booktabs}
% Fix footnotes in tables (requires footnote package)
\IfFileExists{footnote.sty}{\usepackage{footnote}\makesavenoteenv{long table}}{}
\IfFileExists{parskip.sty}{%
\usepackage{parskip}
}{% else
\setlength{\parindent}{0pt}
\setlength{\parskip}{6pt plus 2pt minus 1pt}
}
\setlength{\emergencystretch}{3em}  % prevent overfull lines
\providecommand{\tightlist}{%
  \setlength{\itemsep}{0pt}\setlength{\parskip}{0pt}}
\setcounter{secnumdepth}{0}
% Redefines (sub)paragraphs to behave more like sections
\ifx\paragraph\undefined\else
\let\oldparagraph\paragraph
\renewcommand{\paragraph}[1]{\oldparagraph{#1}\mbox{}}
\fi
\ifx\subparagraph\undefined\else
\let\oldsubparagraph\subparagraph
\renewcommand{\subparagraph}[1]{\oldsubparagraph{#1}\mbox{}}
\fi

% set default figure placement to htbp
\makeatletter
\def\fps@figure{htbp}
\makeatother

\date{}

\usepackage{xr}
\externaldocument{figures}
\externaldocument[S-]{supplementary_materials}
\externaldocument[SM-]{supplementary_materials_methods}

\bibliography{Malaria}
\begin{document}
	

A Meta-analysis of Longevity Estimates of Mosquito Vectors of Disease

Ben Lambert\textsuperscript{1,2}, Ace North\textsuperscript{1} \& H.
Charles J. Godfray\textsuperscript{1}

\textsuperscript{1} Department of Zoology, University of Oxford, South
Parks Road, Oxford OX1 3PS, United Kingdom

Corresponding author:

Phone: 01865 271176

\textsuperscript{2} Present address: MRC Centre for Outbreak Analysis
and Modelling, Infectious Disease Epidemiology, Imperial College London,
London W2 1PG, UK.


\section{Abstract}\label{abstract}

Mosquitoes are responsible for more human deaths than any other
organism, yet we still know relatively little about their ecology.
Mosquito lifespan is a key determinant of transmission strength for the
diseases they vector, but the field experiments used to determine this
quantity -- mark-release-recapture (MRR) studies and wild-caught
dissection of female mosquitoes -- produce estimates with high
uncertainty. In this paper, we use Bayesian hierarchical models to
analyse a previously-published database of 232 MRR experiments and
compile then analyse a database of 131 dissection studies to produce the
first ever species- and genus-level estimates of mosquito lifespan. Due
to the assumptions required to analyse the field data, we term our
estimates lower bounds on lifespan (LBL). Notably, for the major African
malaria vector \emph{Anopheles gambiae s.l.}, we estimate LBLs of 4.5
days (mean estimate; 25\%-75\% CI: 3.8-5.1 days for unfed female
mosquitoes from the MRR analysis) and 9.5 days (mean estimate;
25\%-75\% CI: 5.2-11.0 days from the dissection analysis); and an LBL of
4.3 days (mean estimate; 25\%-75\%
CI: 3.6-4.8 days, only present in the MRR database) for the
predominantly East-African vector \emph{A. funestus s.l}. We estimate
LBLs of 7.0 days (mean estimate; 25\%-75\% CI: 4.5-8.5 days from the
MRR analysis) and 5.0 days (mean estimate; 25\%-75\% CI: 3.5-5.1 days
from the dissection analysis) for \emph{Aedes aegypti}; and 12.1 days
(mean estimate; 25\%-75\% CI: 10.0-13.7 days from the MRR analysis)
for \emph{Ae.} \emph{albopictus} -- the predominant vectors of dengue
fever, chikungunya and Zika. Our estimates indicate that there is
significant variation in lifespan across species, with most variation
explained by differences between genera. In correspondence with
laboratory studies, we estimate that female mosquitoes outlive males by
1.2 days on average (mean estimate; 25\%-75\% CI: 0.3-1.6 days). We
fit models incorporating mosquito senescence to the data, which allows
us to assess evidence for age-dependent mortality in mosquitoes across
different species. We determine that 8 of 33 species included in the MRR
database indicated evidence for senescence, versus only 2 of 25 species
from the dissection database. Our analysis applies a common framework to
the analysis of databases of MRR and dissection-based experiments,
allowing us to produce robust estimates of lower bounds on lifespan. It
also enables us to critically appraise each field method, highlighting a need for alternative field methods for measuring this
important mosquito characteristic.

\section{Author summary}
Mosquitoes transmit some of the most important diseases afflicting humans, with malaria alone killing between 0.4-1.2 million people annually, chiefly children in low-income countries. The transmission strength of these diseases depends critically on the duration of mosquito lifespans and some of the most successful disease control interventions, including insecticide-treated bednets, explicitly target reductions in mosquito longevity. In this study, we conduct meta-analyses of two important classes of field experiments which estimate wild mosquito lifespan: mark-release-recapture studies, where mosquitoes are marked with a dye then released with the number of marked mosquitoes caught monitored over time; and experiments involving dissection of wild-caught females, whose reproductive anatomy is used as a biological clock to determine physiological age. In both analyses, we estimate that most mosquito species live less than 10 days, on average, which suggests that relatively few mosquitoes live sufficiently long to transmit disease. We find evidence of variation in mosquito mortality across species, with the estimates of lifespan obtained from each method largely corresponding for the few species with data from both experiments. Finally, by fitting a range of survival models to the data, we conclude that, for most species, mosquitoes do not experience strong age-related increases in mortality.

\section{Author contributions}
HCJG, AN and BL were involved in conceptualising this study. BL was responsible for data curation and the formal analysis of the data. BL and AN developed the statistical methodology and conducted the investigation. All authors were involved in drafting the original manuscript and revising it.

\section{Keywords}\label{keywords}

mosquitoes, mortality, senescence, mark-release-recapture, vector-borne disease, Bayesian, hierarchical model.

\section{Introduction}\label{introduction}

Some of the most important infectious diseases afflicting humans are
transmitted by mosquitoes \citep{gates2014}, including pathogens such as the causative
agent of malaria that have been associated with humans throughout our
evolutionary history \citep{carter2002evolutionary} as well recently emergent infections such as the
Zika virus \citep{world2016statement}. Most mosquito species have a ``gonotrophic cycle'' involving
successive episodes of vertebrate blood feeding, egg maturation and
oviposition \citep{silver2007mosquito}. In order for a mosquito to transmit a pathogen it must feed
on an infectious person and live long enough to complete at least one
gonotrophic cycle and feed on an uninfected and susceptible individual.
Adult lifespan is thus a critical determinant of the ability of a
mosquito population to allow the persistence of an indirectly
transmitted infection \citep{macdonald1957epidemiology}. Lifespan can of course be straightforwardly
assessed in the laboratory, but it is generally accepted that
measurements under relatively benign laboratory conditions are likely to
have limited relevance in the field, and much effort has been directed
at estimating this parameter in the vector's natural environment \citep{clements1981analysis,guerra2014global}. Most
work has focused on assessing average daily mortality rates, and the
simplest assumption is that these do not vary with mosquito age -- in
this case longevity is simply the reciprocal of mortality. Testing this
assumption and discovering whether mosquitoes senesce or show other
types of age-dependent mortality has also been studied in the field \citep{clements1981analysis,harrington2008age,hugo2014adult}.

There are two main strategies to estimate mosquito mortality rates and
longevity. The first is through mark-release-recapture (MRR)
experiments, a technique that is widely applied to estimate these
parameters in many types of animal. As applied to mosquitoes, insects
are caught in the field or reared in the laboratory and then marked,
typically with fluorescent dust. The mosquitoes are then released into
the field and then efforts are made to recapture them, for example using
human baits or light traps, usually over an extended period of time.
Mortality rates can be statistically estimated from the numbers of
recaptures given certain assumptions \citep{silver2007mosquito}. The main challenges with MRR is
ensuring the marking technique does not affect recapture probability,
and distinguishing mortality from mosquitoes dispersing out of range of
being recaptured. Also, releasing insects that can transmit disease
(especially if this increases ambient population levels) raises
important ethical issues.

The second technique is specific to mosquitoes and makes use of their
gonotrophic cycle \citep{polovodova1949determination,detinova1962age}. The mosquito ovary is made up of ovarioles, each of
which typically produces one egg every gonotrophic cycle. After the egg
passes into the oviduct the distended ovariole does not completely
recover its previous form but a discrete dilation remains which can be
detected by dissecting the female reproductive organs. Data on the
fraction of females that have oviposited provides some information about
mortality rates. However, a skilled dissector can distinguish the number
of dilations from multiple gonotrophic cycles so providing much richer
data on longevity. The challenges of this method include the amount of
time and expertise it takes to collect the data, establishing the
relationships between physiological and chronological time (though the
distribution of the number of gonotrophic cycles wild-caught mosquitoes
have gone through is of direct epidemiological relevance) and the fact
that it only applies to females.

An issue with both methods is that they require logistically difficult
and expensive field campaigns. There is thus value in conducting a
meta-analysis of existing data to explore consistency across studies,
identify correlates of lifespan and to learn lessons for further
studies. Here we analyse data from 232 MRR and 131 dissection studies
using a common statistical methodology. For MRR we make use of a very
valuable database of 394 mosquito studies assembled by Guerra et al.
(2014) while the dissection studies we extracted from the literature
ourselves. We concentrated on the three major genera of mosquito
vectors, \emph{Anopheles, Aedes} (in its traditional sense) and
\emph{Culex}, which constitute the majority of the data.


\section{Results}\label{results}

MRR estimates the length of time a mosquito remains alive and is still
in the area available for recapture. In dissections of females, the majority of ovarioles have fewer dilations than the number of gonotrophic cycles an individual has experienced, also meaning that estimates derived from these data likely understand true physiological age \citep{hugo2008evaluations}. It is unclear which of these methods leads to lower estimates but in both cases we term our estimates lower bounds on lifespan, which we shall refer to as LBL.

\subsection{Lifespan estimates from
MRR}\label{lifespan-estimates-from-mrr}

In 187 of the 230 MRR time
series the estimated LBL was less than 10 days (Fig. \ref{fig:mrr_lifespan_individualEstimates}). The smallest
estimate was 1.1 days for the Asian malaria vector
\emph{Anopheles subpictus s.l.} which is unfeasibly short and almost
certainly reflects dispersal out of the recapture zone or a violation of
the assumptions of our analyses. The longest estimate was 26.9 days for
the temperate species \emph{Aedes simpsoni s.l.} which is a vector of
yellow fever in Africa. There are multiple data sets for the most
important vector species such as \emph{Anopheles gambiae, Aedes aegypti}
and \emph{albopictus} and \emph{Culex tarsalis} all of which show
considerable variation. For example, there are 54 estimates of LBL for
\emph{Ae. aegypti} which range from 2.5 days to 42.1 days with a mean of
11.4 days and coefficient of variation of 0.6 (all estimates are posterior mean). There were significant
differences in LBL amongst species (ANOVA on median LBL controlling for
sex and pre-release feeding: \emph{F}\textsubscript{37,194} = 2.5,
\emph{p} \textless{}0.01; the non-parametric Kruskal Wallace:
\({\chi^{2}}_{38}\), \emph{p}\textless{}0.01).

Since female mosquitoes are most epidemiologically relevant, we start by discussing their estimated lifespan. Also, since the majority of mosquitoes were unfed with blood or sugar prior to release, unless otherwise stated, our estimates represent quantities for unfed mosquitoes. The estimated mean LBL for female mosquitoes for \emph{Culex, Anopheles} and \emph{Aedes} were
2.9, 6.8 and 8.1 days respectively with an overall estimate of 6.0 days
(Fig. \ref{fig:mrr_lifespans}; Table \ref{S-tab:mrr_estimated_lifespans}). Differences between genera were significant (ANOVA on median
LBL controlling for sex and pre-release feeding:
\emph{F}\textsubscript{2,229} = 12.4, \emph{p}\textless{}0.01; Kruskal
Wallace: \({\chi^{2}}_{2} = 30.8,p < 0.01\)). \emph{K}-fold cross
validation suggests that after the effect of genus is accounted for the
incorporation of a species term provides little predictive power (Fig.
\ref{S-fig:mrr_genusTopLevel}; in part explained by the latter model over-fitting the data where
there are few time series per species).

We reasoned that if dispersal out of the recapture area was reducing the
LBL below the true lifespan then there should be a positive correlation
between the spatial extent of the recapture zone and LBL. We found no
such pattern (Fig. \ref{S-fig:mrr_lifeSpanVsRange}), although there was a positive correlation
between LBL and trap density (Fig. \ref{S-fig:mrr_lifeSpanVsTrapDensity}).

The MRR experiments included a mixture of male-only and female-only
releases, and releases of both sexes. We estimated average male and
female LBL at the genus level (Fig. \ref{fig:mrr_sexDifferences_without_sugar_nor_blood}; there were too few studies to
make comparisons at the species level). There was a consistent trend for
females to live longer than males for each of the genera, with the
difference largest for \emph{Aedes} (2.9 days; fraction of pairwise
posterior samples of females versus males where difference was less than
zero, p\textless{}0.01), followed by \emph{Anopheles} (2.2 days; p=0.17)
and \emph{Culex} (0.2 days; p=0.34). Overall, female mosquitoes were
estimated to live 1.2 days longer than males (p=0.10).

The MRR experiments included information on whether mosquitoes were
pre-fed with sugar (41 time series), blood (71), both (4) or
alternatively unfed (116). We estimate that female mosquitoes that were
fed on sugar pre-release lived on average for 1.0 days (posterior mean) longer than those
that were not fed ($p>0.05$; Fig. \ref{S-fig:mrr_female_blood_sugar}; a pattern that was consistent across the
genera). There were insufficient males that were either fed or unfed
with sugar prior to release to make a meaningful comparison. Females
that were blood-fed prior to release on average lived 1.7 days (posterior mean) longer
than those who were not fed for \emph{Aedes} but this trend was reversed
for \emph{Anopheles} meaning that there was little difference overall (0.15 days; posterior mean; $p=0.44$).

To access whether temperature is associated with LBL we used weather
records to calculate average temperatures at the MRR sites (see
Methods). Using both linear and quadratic temperature terms in
regressions, we found no significant relationship between study-site
temperature and LBL (overall or within genus) for the 238 datasets we
analysed (Fig. \ref{S-fig:mrr_temperature}). This result held if, instead of pooling results from
all time series, we considered the four species with the most data
individually (\emph{Ae. aegypti}, \emph{Cx. tarsalis}, \emph{A. gambiae s.l.} and
\emph{A. culicifacies s.l.}; Fig. \ref{S-fig:mrr_ThreeSpeciesVersusTemperature}).

\subsection{Number of gonotrophic cycles estimates from dissection}\label{number-of-gonotrophic-cycles-estimates-from-dissection}

Dissection allows the number of completed gonotrophic cycles to be
counted and from this the mean number of cycles before death was
estimated. Overall, the mean number of cycles completed in a lifetime
was 1.3 (posterior mean; Fig. \ref{fig:dissection_lifetimes_exponential}; Table \ref{S-tab:dissection_estimated_lifespans}) and across the 131 studies, 95\% of
the individual time series estimates were less than 3 (Fig. \ref{S-fig:dissection_individualEstimates_allSpecies}). The
estimated greatest number of cycles was for \emph{Anopheles sergentii}
(3.0 cycles; posterior mean) which is adapted to desert conditions (it
is known as the ``oasis vector'' of malaria) and may have evolved
greater longevity. The important African malaria vector \emph{A.
gambiae s.l.} was estimated to be the second longest living (2.4 cycles;
posterior mean). The smallest estimated mean number of gonotrophic
cycles was for \emph{Anopheles} \emph{bellator} (0.6 cycles; posterior
mean) which transmits malaria in Brazil's Atlantic Forest. There were
significant differences in estimated lifetime gonotrophic cycles amongst
species (ANOVA: \emph{F}\textsubscript{24,106} =2.2, \emph{p}
\textless{}0.01; the non-parametric Kruskal Wallace:
\({\chi^{2}}_{24}\), \emph{p}\textless{}0.01).

The estimated lifetime gonotrophic cycles for the different genera were
\emph{Anopheles,} 1.6; \emph{Culex,} 1.2; \emph{Mansonia}, 1.1; and
\emph{Aedes} 0.8 (Fig. \ref{fig:dissection_lifetimes_exponential}; Table \ref{S-tab:dissection_estimated_lifespans}) and the differences between the genera were
significant (ANOVA: \emph{F}\textsubscript{3,127} =3.4, \emph{p} =0.02;
the non-parametric Kruskal Wallace: \(\chi_{3}^{2} = 21.7\),
\emph{p}\textless{}0.01).

\subsection{Comparison of longevity estimates from two
methods}\label{comparison-of-longevity-estimates-from-two-methods}

Using the data collected from a literature search, we estimated that the first gonotrophic cycle duration had a mean of 4.3 days (std. error: 0.4 days) and, for subsequent cycles, the mean was 3.9 days (std. error: 0.4 days; see SOM). To compare the two methods, we converted numbers of gonotrophic cycles
(physiological age) into lifespan (chronological age) as described in
the SOM using these estimates of gonotrophic cycle duration. Table \ref{S-tab:dissection_estimated_lifespans_chron} provides posterior summaries of chronological for the species and genera in the dissection dataset (see also Fig. \ref{S-fig:dissection_lifetimes_exponential_chron}). For ten species, we had enough data from both species to make a comparison, and there was a positive correlation (not
statistically significant; Pearson correlation $\rho=0.42$, $n=10$, $p=0.23$)
between the two measures (Fig. \ref{fig:comparison}), and in only one case -- for \emph{A.
darlingi} - there was a significant difference in the
time-series level LBLs (Table \ref{S-tab:comparison}).

\subsection{Evidence for age-dependent
mortality}\label{evidence-for-age-dependent-mortality}

The survival model upon which the above analyses are based is the
single-parameter exponential model which assumes an age-invariant
mortality hazard. We also fitted five multi-parameter models that allow,
in different ways, mortality to vary with age. We did this to maximise
our chance of detecting age-varying mortality (though aware of the risks
of false positives with multiple estimations).

In Fig. \ref{fig:mrr_elpd}, we compare the performance of the six models for describing
lifespan in MRR studies of 33 species using K-fold cross-validation. We
categorised the evidence for age-dependent mortality in each species
according to the performance of the five age-dependent models versus the
exponential: `+' indicated that all age-dependent models outperformed
the exponential; `?' indicated that the exponential outperformed one or
more age-dependent models; and `-' indicated that the exponential
performed at least as well as all other models. Overall, we estimated
that there were 8 `+' species, where age-dependent mortality fit the
data better; 11 `?' species where the evidence was mixed; and 14 species
where constant mortality models performed at least as well. The species
where age-dependent mortality best fit the data included the major
vector of dengue fever, Zika and chikungunya, \emph{Ae. Aegypti}. These
studies also tended to include multiple release MRR studies which, on
average, were conducted over a longer period of time than the others,
which may be why we failed to detect age-dependence in the latter (Fig
\ref{S-fig:mrr_mcPowerAnalysis_senescence}).

In Fig. \ref{fig:dissection_elpd}, we compare the performance of the six models for describing
lifespan in dissection studies of 25 species using K-fold
cross-validation, and categorise the evidence in the same way as for the
MRR analysis. By our metric, we determined that there were only two
species with evidence for age-dependent mortality (the major African
malaria vector \emph{A. gambiae s.l.} and \emph{A. minimus}, a malaria
vector in Asia).

Overall, we conclude that there is mixed evidence for age-dependent
mortality from studies of mosquitoes in the field. It is possible that
some of the sampled mosquito species did not live long enough in the
wild to experience physiological decline. A Spearman's rank correlation
test indicated that there was a correlation between the ranked estimated
LBLs of the species and the ranked mean predictive accuracy of
age-dependent models for the MRR analysis ($\rho$=0.19, p=0.01), however was
not significant for the dissection analysis ($\rho$=0.07, p=0.43). Similarly,
a recent study determined that the degree of senescence varies according
to season for semi-wild populations of \emph{Ae. aegypti} (Hugo et al.,
2014), and it is possible that by pooling data from different
geographies and seasons that we failed to detect age-dependent mortality
in some cases.

\subsection{Estimates of the fraction mosquitoes capable of transmitting
disease}\label{estimates-of-the-fraction-mosquitoes-capable-of-transmitting-disease}

We can use the posterior parameter estimates from our Bayesian analysis
to estimate the fraction of mosquitoes that live beyond a certain age.
In order to transmit a disease, a mosquito must live longer than the
length of the intrinsic incubation period (the time taken for a pathogen
ingested in one blood meal to be ready to be transmitted during a future
feeding event). This is a lower bound as it does not include the waiting
time to find a host after feeding or egg maturation. In Fig. \ref{fig:eip}, we plot
the fraction of the mosquito population that pass this threshold using
estimates from both MRR and dissection studies for vector species (see
SOM for references used to identify species as vectors) and their most
significant diseases.

For malaria, estimates of the minimum fraction of the population that
can transmit the disease vary from \textless{}0.1\% for \emph{A.
subpictus} (posterior median; from the MRR analysis, as noted above likely to be due to
the LBL substantially underestimating lifespan) to 52\% (posterior median) for the
drought-adapted and long-lived \emph{A sergentii.} The proportions
surviving long enough to become infectious for the two major African
malaria vectors were: for \emph{A. gambiae s.l.}: 10\% (MRR) and 27\%
(dissection); and for \emph{A. funestus s.l.}: 9\% (MRR). Using the
individual time series estimates, there
evidence for a difference in EIP between the species (Kruskal-Wallis used due to non-normality of data; MRR:
\(\chi_{14}^{2} = 30.2\), \emph{p} \textless{}0.01; dissection: \(\chi_{11}^{2} = 38.9\), p\textless{}0.01).

\emph{Ae. aegypti} and \emph{Ae. albopictus} are the main vectors of
dengue, chikungunya and Zika viruses. Because of their short intrinsic
incubation periods a greater fraction of mosquito potentially live long
enough to transmit diseases (Fig. \ref{fig:eip}), rising to a maximum of 84\% for
\emph{Ae. albopictus} transmitting chikungunya.

\section{Discussion}\label{discussion}

In this study, we applied a Bayesian hierarchical framework to the
analysis of a database of mark-release-recapture experiments and another for mosquito dissection studies to estimate mosquito lifespan.
By applying a single framework, this allows us to effectively synthesise
information from the disparate experiments which, individually, estimate
lifespan with considerable uncertainty. Due to the assumptions required
to analyse the field data, our estimates represent lower bounds on
lifespan (LBL). Across both meta-analyses, the estimated LBLs were
mostly less than 10 days, hinting that only a small proportion of
mosquitoes may live long enough to transmit disease. We determined that
LBL varies across species and genera, although most variance is
explained by genus. The MRR analysis includes experiments conducted on
each sex individually, and we estimate that, on average, males live
shorter lives than females. Pre-release feeding with sugar also
lengthens lifespan across all three genera, although this effect is less
marked than the sex differences. In contrast to a number of lab-based
experiments \citep{yang2009assessing,brady2013modelling}, temperature was not determined to significantly impact
lifespan. By fitting a range of survival models to the data in both
meta-analyses, we could assess evidence for age-dependent mortality.
Overall, we conclude that the evidence is mixed: in the MRR experiments,
in 8 of 33 species we found evidence for mosquito senescence, whereas in
only 2 of 25 species included in the dissection analysis were better fit
by a model incorporating an increasing risk of mortality with age.

MRR experiments are known to produce downwardly-biased estimates of
lifespan. Lab experiments have demonstrated that marking can negatively
impact survival \citep{verhulst2013advances,dickens2014effects}
resulting in artificially depressed survival. MRR studies typically
cannot differentiate between a mosquito dying and dispersal from the
study area meaning that lifespan will be underestimated. In this study,
we found a positive correlation between lifespan estimates and the
density of traps, indicating that better trapping coverage likely raises
estimates towards their real value. We conducted an \emph{in silico}
Monte Carlo study to determine how accurately we could estimate mosquito
lifespan given study parameters in an ideal MRR experiment, where the
assumptions of no emigration and harmless marking are fully satisfied
(see SOM for full details). This work indicated that for many of the
experiments, the short study lengths or typical numbers of mosquitoes
released, results in considerable uncertainty in lifespan estimates (Fig.
\ref{S-fig:mrr_mcPowerAnalysis}). This indicates that statistical power can be substantially
increased by pooling data across experiments as we did using a Bayesian
hierarchical model.

The key assumptions of dissection based methods to determine
chronological age are: (i) physiological age can be accurately
determined by dissection of female specimens (unlike MRR, this method
can only be applied to one sex), (ii) the relationship between
physiological and chronological age is known, (iii) the population being
sampled is in equilibrium (recruitment matches mortality) and (iv)
individual mosquitoes can be randomly sampled from the population. The
reliability and accuracy of dissection has been questioned. The
objections include the impracticality of dissecting more than a small
proportion of ovarioles \citep{hoc1995ovariole}, particularly in African
vector species \citep{gillies1965study}, the related issue of locating
ovarioles whose count of dilations represents true physiological age
\citep{fox1994dilatations}, and the variation in numbers of ovariolar
dilations for mosquitoes of the same, known, physiological age \citep{kay1979age,russell1986population,hugo2008evaluations}. Indeed there is considerable
uncertainty concerning the fundamental question of how dilations in
ovarioles form in the first place. Whilst the `Old School' of thought (a
term coined by Fox and Brust, 1994) headed by Polovodana \citep{polovodova1949determination} and Detinova \citep{detinova1962age} considers dilations to result from
normal oogenesis, a `New School' headed by Lange and Hoc \citep{lange1981abortive} has challenged this assertion. The New School believe that only
abortive oogenesis results in follicular dilations because normal
oogenesis destroys the sack-like structures \citep{fox1994dilatations}. This
means that Polovodana's method requires dissecting large numbers of
ovarioles to uncover those with the most dilations, where abortive
oogenesis has occurred in each gonotrophic cycle. They deem these
ovarioles `diagnostic' since only in these cases the number of dilations
equals the number of gonotrophic cycles that have occurred. As a
mosquito ages, the number of diagnostic ovarioles diminishes, since the
random occurrence of normal oogenesis in a particular ovariole means its
dilation count does not equal the number of gonotrophic cycles
undertaken. This increased difficulty of finding diagnostic ovarioles as
a mosquito ages would elevate the chance of age `hypodiagnosis' for
older specimens \citep{fox1994dilatations}, and likely biases lifespan
estimates downwards. The difficulty of locating diagnostic ovarioles has
been investigated using lab populations of \emph{Culex} and \emph{Aedes}
mosquitoes by Hugo et al. (2008), who conclude that only a small
percentage of ovarioles are diagnostic. The exchange rate between
physiological age and chronological age is the duration of gonotrophic
cycles. Two methods are commonly used to estimate the duration of
gonotrophic cycles: MRR studies (see, for example, \cite{gillies1965study}), where marked mosquitoes are recaptured and dissected to determine
the number of gonotrophic cycles occurring since release; and
laboratory-based observations of colonies of (typically) wild-caught
females, or their progeny (see, for example, \cite{afrane2005effects}).
Whilst it is unclear how each method could bias estimated gonotrophic
cycle duration, in our analysis, laboratory-based studies indicated a
longer gonotrophic cycle (Fig. \ref{S-fig:dissection_gonotrophicCycleRaw_MRRVsLab}). The distributions we used to
convert physiological age into calendar age were calculated by pooling
data across both approaches, to incorporate uncertainty from both
experimental procedures. It is possible, however, that this aggregate
approach may induce biases in estimates and an approach more entrenched
in experimental knowledge would fare better. If a population of
mosquitoes is shrinking, this leads to a relative under-abundance of
young mosquitoes, and a flattening of the survival curve, resulting in
over-estimates of lifespan. For stable populations, periods when
shrinking occurs must result in equal changes in the population size
compared to those when it expands. If mosquito collections occur with
equal frequency in each of these two modes, then aggregating the data
across all sampling times and estimating a single model, as we do here,
should yield an approximately unbiased estimate of lifespan. The
additional uncertainty of a fluctuating population size, however, could
lead us to understate the uncertainty in estimates. Field entomologists
have challenged the assumption of random sampling the mosquito
population, although there are conflicting opinions as to whether this
results in a relative paucity \citep{gillies1965study} or abundance
\citep{clements1981analysis} of nulliparous individuals. In our
database, there are cases where there was an obvious deficit of
nulliparous individuals, which has previously been ascribed to the
differing distribution of resting females between indoor and outdoor
traps \citep{detinova1962age,clements1981analysis}. We chose to not
include those counts of nulliparous individuals in our analysis where
their number was less than 90\% of the uniparous. Whilst we see no
obvious differences in lifespan according to collection method (data not
shown) or location, it is possible that the assumption of random
sampling is violated, although the directionality of the bias induced by
this is unclear. Overall, the assumptions underpinning estimates from
dissection studies indicate that our estimates represent lower bounds on
lifespan. The alternative dissection-based approach of Detinova
\cite{detinova1962age}, based on dichotomous categorisation of female mosquito
specimens as `parous' or `unparous' relies on fewer assumptions, and is
widely used. Further work examining parity rates in field specimens may
be fruitful although, in principle, it offers less information on the
age structure of a population than Polovodova's approach.

By applying a common method to analysing all studies in our databases,
it is possible that we may have missed patterns of mortality that would
have been evident from using a more bespoke approach. As our \emph{in
silico} analysis of MRR experiments indicates, however, the
overdispersed data from single experiments results in high measurement
error (Fig. \ref{S-fig:mrr_mcPowerAnalysis}). By applying different methods to each study, this
could lead us to falsely detect patterns when none are present, and we
prefer a pooled approach.

The different nature of the assumptions of each of the two methods means
they offer complimentary information on mosquito survival. We also note
that Polovodova's dissection-based studies require specialised expertise
which will often be unavailable, whereas MRR methods can more readily be
used. Furthermore, most if not all dissection methods that have been
used previously are only applicable to female mosquitoes, whereas MRR
can be applied to either sex and can additionally be used to determine
other ecological parameters (for example, population size and
dispersal). Although dissection data gives detailed of age-structure, we
thus foresee a continued reliance on MRR experiments in field
entomological experiments. Efforts to use both approaches concurrently
will be particularly useful and will allow quantification of the biases
induced by the assumptions of each. Similarly, MRR experiments releasing
large numbers of marked mosquitoes and recording
spatiotemporally-disaggregated captures of wild and re-caught marked
mosquitoes will continue be useful in estimating lifespan and dispersal.

To compare estimates of lifespan derived from MRR with those from
dissection-based methods, we display the estimates of lifespan from
those ten species occurring in both databases in a single plot (Fig. \ref{fig:comparison}). In is reassuring that there is correlation between estimates from both approaches, although the small sample size likely hindered our ability
to determine statistical significance. In both cases, we estimate that
\emph{A. sergentii} was amongst the longest lived of the anopheline
species with an LBL of 12.4 days (mean estimate;
25\%-75\% CI: 5.9-13.8 days) from the MRR analysis and 11.9 days (mean
estimate; 25\%-50\% CI: 7.6-14.0 days) from the analysis of dissection
studies. This species is a major vector of malaria in the Sahara \citep{sinka2010dominant}, where to act as a disease vector it must persevere
through these hard conditions. It is reasonable to hypothesise that this
species should live longer than those in environments where the
potential for blood-feeding and oviposition is greater. The species with
the greatest discrepancy in the estimates was the major African malaria
vector \emph{A. gambiae s.l.}, where we estimated LBLs of 4.5 days
(mean estimate; 25\%-75\% CI: 3.8-5.1 days for unfed female) from the
MRR analysis and 9.5 days (mean estimate; 25\%-75\% CI: 5.2-11.0) from
the dissection analysis. Across genera, the greatest discrepancy in estimates was for
\emph{Aedes}, where the estimates from the MRR studies (8.1 days) are considerably longer than those of dissection-based studies (3.5 days).
This was followed by \emph{Culex} (a posterior mean of 2.9 days from the
MRR versus 4.9 days from the dissection analysis) with the smallest
discrepancy for \emph{Anopheles} (6.8 versus 6.4 days). Across all studies we estimate
from the MRR analysis that mean mosquito lifespan is 6.0 days versus 5.5
days from the dissection-based studies. Some of the differences in these
group-level estimates between the two approaches is likely due to
environmental and genetic differences between mosquitoes in the
experiments that were analysed in each meta-analysis. However, we
believe that part of the discrepancy can be explained by the
methodological differences in approaches. We speculate that differences
in dispersal rate can explain some of the discrepancy. Both
\emph{Anopheles} and \emph{Culex} mosquitoes are generally thought to
fly farther during their lifetimes than \emph{Aedes} {[}Charles, do you
have a reference here?{]}, meaning that the estimates from MRR-based
approaches will be most downwardly-biased for these genera. This is
supported by our results since the dissection-based estimates
(themselves not reliant on assumptions about dispersal) are similar or exceed the MRR
estimates for \emph{Anopheles} and \emph{Culex} mosquitoes, but not for
\emph{Aedes}.

It is widely believed mosquitoes live artificially long under the benign
conditions of the laboratory. We find it informative to consider
estimates of lifespan derived from observations of such populations as
they constitute an upper bound on the lifespan of wild populations.
Also, since the numbers of mosquitoes involved in large cage experiments
often numbers in the thousands, these estimates have lower uncertainty
than those from field experiments although are typically conducted on
highly inbred mosquito strains. \cite{styer2007mosquitoes}, using colonies of
45,054 female and 55,997 male \emph{Ae. aegypti}, determined that
females lived nearly twice as long as males; the median lifespan was
estimated as 31.69 $\pm$ 0.06 days for females and 16.39 $\pm$ 0.03 days
for males. A similar study by \cite{dawes2009anopheles} with a lab colony of
over 1000 female \emph{A. stephensi} found similar estimates for median
lifespan (31-42 days). These estimates are many multiples of the average
estimates that result from our analysis of field data which, as
discussed, represent lower bound estimates. Without an unbiased method
to measure mosquito lifespan, however, it is difficult to quantify and
explain the gap that exists between field and laboratory lifespans. The
development of additional methods to estimate mosquito age, such as
`Near-Infrared Spectroscopy' \citep{mayagaya2009non,sikulu2011evaluating,lambert2018monitoring} if they are proven to work in the field,
may be of considerable worth here.

We conducted a power analysis of MRR experiments to determine
whether typical experimental characteristics could detect senescence.
Here we calculated the power of a maximum likelihood estimator of the
`senescence parameter' $\beta$ of the Gompertz survival function (see Table
\ref{SM-tab:mrr_survivalDescription}) for case study populations with three different levels of
senescence (Fig. \ref{S-fig:mrr_mcPowerAnalysis_senescence}A). This analysis indicated that power to detect
senescence strongly depends on study length (Fig. \ref{S-fig:mrr_mcPowerAnalysis_senescence}B) but is
insensitive to release size (Fig. \ref{S-fig:mrr_mcPowerAnalysis_senescence}C). Clements and Patterson (1981)
conducted a meta-analysis of MRR and dissection-based field experiments
and found evidence of an increasing risk of mortality hazard with age
that is similar in magnitude to that of the `mild' case considered
above. For this case, detecting senescence with a power of 80\% requires
a study length of at least 18 days. Since the median study duration for
experiments included in our analysis was 10 days (Table \ref{SM-tab:mrr_IndividualData}) this could
partly explain our failure to detect senescence at the species level. A
number of experiments have found evidence of age-dependence in
laboratory populations \citep{styer2007mosquitoes,dawes2009anopheles}.
However, the artificially benign environment of the laboratory means
mosquitoes live considerably longer than in the wild, where they may die
because of exogenous factors, before the effects of physiological
decline have had time to manifest. Field experiments have also found
evidence for age-dependent mortality. Harrington et al. (2008) conducted
a field experiment where mosquitoes reared under laboratory conditions
were marked and released at different ages. Analysis of the resultant
MRR time-series indicated that mosquito mortality increases with age at
release. It is possible, however, that this field experiment suffers
from the same biases as laboratory-based approaches, because the
released mosquitoes were often of ages considerably higher (up to 20
days) than typical estimates of wild mosquito lifespan.

As ethical concerns of contributing to disease burden are more often
considered, it is now less common for MRR experiments to release female
mosquitoes versus males than historically (Fig. \ref{SM-fig:mrr_sexReleasesOverTime}). Our analysis
indicates that females outlive male mosquitoes by approximately 1.2 days
(Fig. \ref{fig:mrr_sexDifferences_without_sugar_nor_blood}), meaning that differences between the sexes may exist for
other ecological parameters determinable by MRR. This suggests that
continued field entomological work on contained releases of mosquitoes
in semi-field sites or large microcosms may be a valuable source of
information on female mosquito ecology.

Our estimates of LBL indicate that mosquitoes that were sugar-fed prior
to release lived on average 0.7 days longer than those that were unfed (Fig. \ref{S-fig:mrr_female_blood_sugar})
suggesting the potential value of this underappreciated aspect of the
mosquito ecology to the insects. It may also partly explain the recent
successes in the use of Attractive Toxic Sugar Baits as a vector
control intervention \citep{muller2008decline,muller2010effective,muller2010field,muller2010successful,beier2012attractive}. More research is
needed, however, to identify the sugar-feeding frequency and food
sources for wild populations.

There is evidence mainly from laboratory studies that temperature
modulates mosquito ecology and behaviour \citep{yang2009assessing,brady2013modelling,murdock2012complex,beck2013effect}. The
locations and times of year over which the MRR studies were conducted
encompassed a large range of average air temperatures, from
approximately 10 \textsuperscript{o}C to 35 \textsuperscript{o}C and,
within this, we determined no relationship between lifespan and
temperature across all time series (Fig. \ref{S-fig:mrr_temperature}) or, for any of the species
with the most data (Fig. \ref{S-fig:mrr_ThreeSpeciesVersusTemperature}). It is possible that by considering a raw
average of air temperature across the month, this ignored, more complex, interactions between temperature and lifespan. It is also possible that by ignoring the effects of
rainfall (the historical data on rainfall is less likely to be reliable
for a given location), that this masked a more complex interaction
between longevity and temperature. The observed laboratory relationship
between lifespan and temperature, however, may not be as robust in the
field if mosquitoes adjust their behaviours (such as, by seeking shade)
in reaction to changes in temperature. More work exploring the
relationship between mosquito ecology and temperature in semi-field
experiments may be useful in probing these interactions further.


In this work, we have used modern statistical methods to synthesise precious field data conducted by entomologists past and present, to produce lower bound estimates of mosquito lifespan. The importance of vector mortality for disease transmission has long been recognised, however,
since even before 1957, when George Macdonald formulated the
now famous Ross-Macdonald equation of R\textsubscript{0} for malaria. Indeed, the recent declines in malaria prevalence in Sub-Saharan Africa were likely due to upscaling of interventions (insecticide-treated bednets and indoor residual spraying) that aim to reduce mosquito lifespan \citep{bhatt2015effect}. Worryingly, resistance to pyrethroids, the only class of insecticide used in current insecticide-treated bednets and likely the only product to come to market in the near
future, has been determined to be widespread and increasing in intensity
across Sub-Saharan Africa \citep{world2018global}. This alarming trend
highlights the need for continued MRR and dissection-based studies to
monitor the effectiveness of bednets and determine whether more
expensive alternatives, such as nets incorporating piperonyl butoxide be
deployed. It also emphasises the need for investment in new tools for
real time monitoring of mosquito populations. In recent years, considerable funding has been allocated to molecular and genomic research
into mosquitoes that strengthens existing interventions and suggest
novel control strategies. Without commensurate funding allocated to
applied vector ecology, our lack of knowledge in this area threatens our
opportunity to capitalise on molecular advances and potentially hinders
our ability to control of mosquito-borne disease.

\section{Methods}\label{methods}

In recent years many important vectors of disease have been shown to be
complexes of closely related species, biotypes or forms that cannot
be distinguished morphologically (for example the morphospecies
\emph{Anopheles gambiae sensu lato} is now separated into the widespread
\emph{gambiae, coluzzii, arabiensis} and a number of more local
species). As the majority of studies analysed here took place before
molecular techniques allowed these taxa to be separated we work here
chiefly with morphospecies.

\subsection{Mark-release-recapture}\label{mark-release-recapture}

Data from MRR experiments in the Guerra et al. (2014) database were
examined and those with fewer than six recaptures and species with only
a single MRR study were excluded for the hierarchical analysis. Of the
232 data sets, 179 involved only females, 35 males, and 18 both sex
releases. For 102 data sets the age of the released mosquitoes was known
(the average age of released mosquitoes was 4.0 days) while in the other
cases it was unknown or unrecorded; in these cases we assumed the
mosquitoes were newly emerged at the time of release and return to this
assumption later. See Table \ref{SM-tab:mrr_aggregateData} for a summary of other data characteristics.

We analysed all MRR experiments within the same statistical framework
(for full details see the Supplementary Online Material (SOM)). In the
simplest case \emph{N\textsubscript{R}} mosquitoes are released on day
zero and the probability that they remain in the recapture area until
day \emph{t} is \emph{S}(\emph{t}) when they are recaptured with
probability $\psi$. We model the number of mosquitoes recaptured on
day \emph{t} using a negative binomial sampling model with mean
\(\left( N_{R} - Y\left( t - 1 \right) \right)S\left( t \right)\psi\),
where \(Y\left( t - 1 \right)\) is cumulative captures before day
\emph{t}, and shape parameter $\kappa$. The negative binomial has been
used previously in analyses of mosquito count data \citep{service1971studies,nedelman1983negative} because of its ability to
represent temporal over-dispersion in recaptures most likely caused by
variable weather. A slight modification was required for studies with
multiple releases (see SOM).

The simplest model for $S(t)$ assumes there is a constant
probability ($\lambda$) that a mosquito dies or leaves the recapture area
so that the numbers remaining after time \emph{t} are given by the
exponential distribution, $\text{exp}( -\lambda t )$.
We utilised this form extensively but in testing for senescence used
five other models where $\lambda(t)$ varies with time so that,

\begin{longtable}[]{@{}ll@{}}
	\(S\left( t \right) = e^{- \int_{0}^{t}{\lambda\left( \tau \right) \mathrm{d}\tau}}.\)
\end{longtable}

Details of the five models (Gompertz, Weibull, Gompertz-Makeham,
Logistic and Logistic-Makeham), which vary in their ability to detect
different forms of age-dependent mortality, are given in the SOM. Using
multiple different types of models increased our chances of detecting
senescence though, as discussed below, also increases the likelihood of
false positives.

Parameters were estimated using Bayesian techniques with relative
uninformative priors for $\kappa$ and the parameters of
$\lambda(t)$, but assuming a prior for $\psi$ indicating a low
recapture probability (bounded in part by knowledge of the maximum daily
recapture rates; see SOM). We used a Bayesian hierarchical model to estimate
distributions of lifespan at the species and the genus levels, and
across the complete data set. This procedure assumes that there is a
distribution of lifespan parameters for each species from which those
governing individual MRR time series are sampled, and similarly a
distribution at the genus level from which those for individual species
are derived (rather akin to random effects in classical statistics).
Within this framework we can also allow the parameters for individual
time series to be influenced by co-variates such as differences in
experimental methodology. As in the estimation of the parameters of the
individual experiments, relative uninformative priors were set for the
parameters of the hierarchical models except for $\psi$ where again a
distribution representing low recapture probabilities was assumed.
Posterior distributions were derived using Markov Chain Monte Carlo
(MCMC) methods with convergence assessed using the \(\hat{R}\) statistic
\citep{gelman1992inference}. The predictive power of the model was assessed
using \emph{K}-fold cross validation which tests the ability of the
model fitted to part of the data to predict the rest using multiple
different partitions. Further details of the prior specification,
fitting and validation through posterior predictive checks \citep{lambert2018student} are given in
the SOM.

Two studies of \emph{Anopheles balabacensis} reported capture rates
increasing with time, presumably reflecting a violation of our
assumption of constant recapture probabilities. We omitted this species
from the analysis.

The Guerra et al., (2014) database included the latitude and longitude of each
study along with the date when the study began. We used this information to find
estimates of the air temperature for each study using the European Centre for Medium
Range Weather Forecasts' ERA Interim Daily historical database. For each
study we calculated the mean monthly temperature across a spatial area
of (latitude $\pm$ 1 degree, longitude $\pm$ 1 degree), for the month at which
each study was carried out. The records for this database begin in 1979,
which pre-dates the study date for 65 of our 232 MRR time-series. For
these time-series, we chose to estimate the air temperature by an
average of the corresponding monthly temperatures over the years
1979-89.

\subsection{Dissection}\label{dissection}

Studies using dissection to estimate mosquito longevity were located in
literature databases using relevant keyword, citation and author
searches, and by checking previous studies cited by the papers located
(see SOM). The list of studies located with associated metadata is
available as a Supplementary Online File.

Most dissection studies recorded the distribution of the number of
gonotrophic cycles in mosquito samples collected over a specific period
of time. Overall, we found 568 physiological age cross-sections at recorded distinct times in 72 published articles. Our statistical approach relies on steady recruitment to the adult mosquito population. To guard against the effect of fluctuating population sizes on our analysis, we aggregated the data at a given location across cross-sections taken at different times. We further omitted time series with fewer than 100 mosquitoes and for
species with only one data set leaving 131 studies of mosquitoes in the
genera \emph{Anopheles, Aedes}, \emph{Culex} and \emph{Mansonia}.

The data which we use provides measures of the age distribution of mosquitoes within each investigated population. By assuming that the population sizes were approximately fixed throughout the period of investigation, this allows us to estimate mean lifespan using a statistical model of mortality incorporating the probability of mosquito capture. We modelled the number of mosquitoes found by dissection to be of age
\emph{a} using the negative binomial distribution with mean
$\Psi S(a)$ and shape parameter $\kappa$, where
$\Psi$ is the product of the recruitment rate of adult mosquitoes,
which we assume is constant over time, and the probability of being
captured for dissection, and \emph{S}(\emph{a}) is the probability of
surviving until age \emph{a}. We used the number of females that have
yet to lay eggs (nulliparous) to estimate the recruitment rate as
described further in the SOM. Initial examination revealed that in some
data sets the number of nulliparous females was anomalously low,
something that has been noticed before \citep{gillies1965study}. As some
studies have suggested that the first gonotrophic cycle tends to be
longer than the subsequent ones, this is probably due to differences in
capture probability. In data sets where the fraction of nulliparous
females was less than 90\% the uniparous (completed on gonotrophic
cycle) we excluded the nulliparous observation. Data was analysed using a Bayesian framework as with the MRR data with
minor differences in the specification of the priors (see SOM). 

To compare lifespan estimates from dissection and MRR studies we need to
convert physiological age (the number of gonotrophic cycles) into
chronological age. Using a literature search and a review by Silver
(2007) we found 79 estimates in 42 published articles. Most estimates
were obtained by dissecting females recaptured in MRR studies or by
observations in the laboratory, the latter tending to give longer
durations. Studies differed greatly in how (if at all) they represented
uncertainty in their estimate of the duration of the gonotrophic cycle.
Where confidence limits were given we treated these as the relevant
quantiles of a normal distribution, where a range was stated (e.g. ``4-6
days'') we interpreted the bounds as the 2.5\% and 97.5\% quantiles of a
normal distribution, and where a single figure was quoted we assumed
this was the mean this distribution. Using the quantiles of the normal
distribution, we estimated its mean and standard deviation by regression
(see SOM). Initially we calculated distributions of gonotrophic cycle
lengths at the species and then genus levels, but because of the paucity
of data for many species and the lack of significant differences we
aggregated the data into a single distribution. We converted
physiological age to chronological age by sampling from this
distribution to obtain a particular gonotrophic cycle length for each
mosquito (we also explored sampling from this distribution to obtain the
duration of \emph{each} gonotrophic cycle which increased the
uncertainty in lifespan estimate but did not affect any of the
conclusions).

\section{Acknowledgements}\label{acknowledgements}

The authors would like to thank the following for useful conversations
throughout the course of this work: Austin Burt, Mike Bonsall, Thomas Churcher, Steve
Lindsay and Ellie Sherrard-Smith.

\printbibliography

\end{document}
